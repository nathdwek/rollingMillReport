\section{Description of the plant}
The goal of this project is to control the rolling of a metallic strip using a rolling mill. The plant is composed of two DC motor driven single rolls which unwind and wind up the sheet of metal. Between those two, the strip passes through a pair of rolls driven by a third DC motor, in order to have its thickness reduced and made uniform.

The actuators of the plant are the three DC motors, which are controlled through their armature current. The speed of those motors are measured using three velocity sensors. There are also two traction sensors to measure the tension in the strip left and right of the middle pair of rolls. Finally, a thickness sensor measures the thickness of the metallic strip after it has been rolled.

The first requirement is to control the traction of the metallic strip. If this is achieved, a more advanced requirement is to control the thickness of the metallic strip.

\section{Broad design of the controller}
The plant will be controlled using a numerical controller implemented in matlab. To set this up, eight ADC and two DAC ports are available, along with second order Butterworth filters with various $\tau$ and an analog computer.

Intuitively, we know that the traction of the strip between two rolls will probably mostly depend on the difference of speed between the rolls. Similarly, we also know that the thickness reduction at the middle pair of rolls will probably mostly depend on the difference between the traction of the strip that is fed into the pair of rolls and the traction of the sheet that is pulled out of it. Knowing this, we propose to control this process using a cascade controller.

First, we control the speed of the DC motors using the current as input. Then, we control the traction of the metallic strip using the speed of the motors as input. Finally, we control the thickness of the metallic sheet using the tractions of the strips as input.

Moreover, since we know that the traction will depend on a \emph{difference} of speeds, we also propose to simplify this scheme by modulating the speed of only one motor during the operation. One motor will be designated as ``master'': it will be controlled with a constant reference with the only goal of spinning at the setpoint speed as steadily as possible. The other will be designated as ``slave'': its reference will vary during operation in order to control the traction of the strip and it should have adequate transient response properties.

\section{Structure of this report}
