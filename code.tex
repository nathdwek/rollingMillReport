\begin{matlabcode}
%controller.m
function [time, reference, output, input, innerTractionRefArray, rSpeedRefArray] = controller()

  %%%%%%%%%%%%%%%%%%%%%%%%%%%%%%%%%%%%%%%%%%%%%%%%%%%%%%%%%%%%%%%%%%%%%%%%%%
  %Control System Design Lab: Overall Controller Loop
  %%%%%%%%%%%%%%%%%%%%%%%%%%%%%%%%%%%%%%%%%%%%%%%%%%%%%%%%%%%%%%%%%%%%%%%%%%
  %%
  %%%%%%%%%%%%%%%%%%%%%%%%%%%%%%%%%%%%%%%%%%%%%%%%%%%%%%%%%%%%%%%%%%%%%%%%%%
  %Setup
  %%%%%%%%%%%%%%%%%%%%%%%%%%%%%%%%%%%%%%%%%%%%%%%%%%%%%%%%%%%%%%%%%%%%%%%%%%

  T_S = 0.01;%Set the sampling time.


  function [nSamples, time, reference, output, input] = referenceGenerator()
    %REF(1,:) = traction
    %REF(2,:) = master speed

    %traction
    EXP_LENGTH = 50;%sec
    TRACTION_REF = 3;
    TRAC_SETUP_TIME = 8;

    tracRamp = 0:T_S*TRACTION_REF/TRAC_SETUP_TIME:TRACTION_REF;
    reference(1,:) = [tracRamp TRACTION_REF*ones(1, EXP_LENGTH/T_S - length(tracRamp))];

    %master
    SETUP_TIME = 8;%seconds
    SETUP_POINT = 2;%V Left Motor

    ramp = 0:T_S*SETUP_POINT/SETUP_TIME:SETUP_POINT;
    rampLength = length(ramp);

    reference(2,:) = [ramp ones(1,length(reference(1,:))-rampLength)*SETUP_POINT];

    nSamples = length(reference);
    time=0:T_S:(nSamples-1)*T_S;
    output = zeros(2, nSamples);
    input = zeros(8, nSamples);
  end
\end{matlabcode}
\clearpage
\begin{matlabcode}
  %%
  %%%%%%%%%%%%%%%%%%%%%%%%%%%%%%%%%%%%%%%%%%%%%%%%%%%%%%%%%%%%%%%%%%%%%%%%%%
  %Controllers
  %%%%%%%%%%%%%%%%%%%%%%%%%%%%%%%%%%%%%%%%%%%%%%%%%%%%%%%%%%%%%%%%%%%%%%%%%%

  function current = lmController(speedRef, measuredSpeed)
    if speedRef < 0
      FRICTION_COMPENSATOR  = -2.7;
    else
      FRICTION_COMPENSATOR  = 2.7;
    end
    KP = 2;
    KI = KP/3.642;

    persistent errorIntegral;
    if isempty(errorIntegral)
      errorIntegral = 0;
    end

    error = speedRef - measuredSpeed;
    errorIntegral = errorIntegral + error*T_S;

    current = FRICTION_COMPENSATOR + KP*error + KI*errorIntegral;
  end

  function current = rmController(speedRef, measuredSpeed)
    if speedRef < 0
      FRICTION_COMPENSATOR  = -2.1;
    else
      FRICTION_COMPENSATOR  = 2.1;
    end
    KP = 3;

    error = speedRef - measuredSpeed;
    current = FRICTION_COMPENSATOR + KP*error;
  end

  function speedRef = innerLoopSpeed(traction, innerTractionRef)
    K = -0.123;
    speedRef = K*(innerTractionRef-traction);
  end

  function innerTractionRef = outerLoopTraction(traction, tractionRef)
    persistent errorIntegral;
    if isempty(errorIntegral)
      errorIntegral = 0;
    end
    K = 2;
    ZERO = 1.9;
    KI = ZERO*K;
    error = tractionRef - traction;
    errorIntegral = errorIntegral + error*T_S;
    innerTractionRef = K*(error) + KI*errorIntegral;
  end

\end{matlabcode}
\clearpage
\begin{matlabcode}
  %%
  %%%%%%%%%%%%%%%%%%%%%%%%%%%%%%%%%%%%%%%%%%%%%%%%%%%%%%%%%%%%%%%%%%%%%%%%%%
  %Main Loop
  %%%%%%%%%%%%%%%%%%%%%%%%%%%%%%%%%%%%%%%%%%%%%%%%%%%%%%%%%%%%%%%%%%%%%%%%%%
  [nSamples, time, reference, output, input] = referenceGenerator();
  %precompute the reference and generate arrays of according size
  innerTractionRefArray = zeros(1,nSamples);
  rSpeedRefArray = zeros(1,nSamples);
  i=1;
  while i < nSamples
    tic %Begins the first strike of the clock.
    [input(1,i), input(2,i), input(3,i), input(4,i), input(5,i), ...
      input(6,i), input(7,i), input(8,i)]  = anain; %Acquisition of the measurements.

    rmVelocity = input(5,i); %saving the inputs in a variable for ease of working
    lmVelocity = input(4,i);
    lTraction = input(3,i);
    rTraction = input(2,i);

    if(lTraction > 6 || rTraction > 6) % safety measures if traction is to high
      lmCurrent = 0;
      rmCurrent = 0;
      i = nSamples + 1;
    else
      lmCurrent = lmController(reference(2,i), lmVelocity);

      %Cascade loops
      innerTractionRef = outerLoopTraction(rTraction, reference(1,i));
      rSpeedRef = innerLoopSpeed(rTraction, innerTractionRef);
      rmCurrent = rmController(rSpeedRef, rmVelocity);
      innerTractionRefArray(i) = innerTractionRef;
      rSpeedRefArray(i) = rSpeedRef;
    end

    output(2,i) = rmCurrent;
    output(1,i) = lmCurrent;
    anaout(lmCurrent, rmCurrent);

    if toc > T_S
      disp('Sampling time too small');%Test if the sampling time is too small.
    else
      while toc <= T_S
        %Does nothing until the second strike of the clock reaches the sampling time set.
      end
    end
    i=i+1;
  end
end
\end{matlabcode}
\clearpage
\begin{matlabcode}
%main.m
openinout;
[time, reference, output, input, innerTractionRef, rSpeedRef] = controller();
anaout(0,0);
closeinout;
%%
%%%%%%%%%%%%%%%%%%%%%%%%%%%%%%%%%%%%%%%%%%%%%%%%%%%%%%%%%%%%%%%%%%%%%%%%%%
%Plots
%%%%%%%%%%%%%%%%%%%%%%%%%%%%%%%%%%%%%%%%%%%%%%%%%%%%%%%%%%%%%%%%%%%%%%%%%%
figure %Open a new window for plot.
plot(time,reference(1,:), time, input(2,:), time,...
  input(4,:), time, input(5,:), time, rSpeedRef); %Plot the experiment (input and output).
legend('traction reference','traction','master speed',...
  'slave speed', 'slave speed reference');
title('Reference and output values')
xlabel('Time [s]');ylabel('Measurement [V]');

figure;
plot(time, output(1,:), time, output(2,:));
legend('Master motor current', 'Slave motor current');
title('Actuators input')
xlabel('Time [s]');ylabel('Value [V]');
\end{matlabcode}
